% Options for packages loaded elsewhere
\PassOptionsToPackage{unicode}{hyperref}
\PassOptionsToPackage{hyphens}{url}
%
\documentclass[
]{article}
\usepackage{amsmath,amssymb}
\usepackage{iftex}
\ifPDFTeX
  \usepackage[T1]{fontenc}
  \usepackage[utf8]{inputenc}
  \usepackage{textcomp} % provide euro and other symbols
\else % if luatex or xetex
  \usepackage{unicode-math} % this also loads fontspec
  \defaultfontfeatures{Scale=MatchLowercase}
  \defaultfontfeatures[\rmfamily]{Ligatures=TeX,Scale=1}
\fi
\usepackage{lmodern}
\ifPDFTeX\else
  % xetex/luatex font selection
\fi
% Use upquote if available, for straight quotes in verbatim environments
\IfFileExists{upquote.sty}{\usepackage{upquote}}{}
\IfFileExists{microtype.sty}{% use microtype if available
  \usepackage[]{microtype}
  \UseMicrotypeSet[protrusion]{basicmath} % disable protrusion for tt fonts
}{}
\makeatletter
\@ifundefined{KOMAClassName}{% if non-KOMA class
  \IfFileExists{parskip.sty}{%
    \usepackage{parskip}
  }{% else
    \setlength{\parindent}{0pt}
    \setlength{\parskip}{6pt plus 2pt minus 1pt}}
}{% if KOMA class
  \KOMAoptions{parskip=half}}
\makeatother
\usepackage{xcolor}
\usepackage[margin=1in]{geometry}
\usepackage{color}
\usepackage{fancyvrb}
\newcommand{\VerbBar}{|}
\newcommand{\VERB}{\Verb[commandchars=\\\{\}]}
\DefineVerbatimEnvironment{Highlighting}{Verbatim}{commandchars=\\\{\}}
% Add ',fontsize=\small' for more characters per line
\usepackage{framed}
\definecolor{shadecolor}{RGB}{248,248,248}
\newenvironment{Shaded}{\begin{snugshade}}{\end{snugshade}}
\newcommand{\AlertTok}[1]{\textcolor[rgb]{0.94,0.16,0.16}{#1}}
\newcommand{\AnnotationTok}[1]{\textcolor[rgb]{0.56,0.35,0.01}{\textbf{\textit{#1}}}}
\newcommand{\AttributeTok}[1]{\textcolor[rgb]{0.13,0.29,0.53}{#1}}
\newcommand{\BaseNTok}[1]{\textcolor[rgb]{0.00,0.00,0.81}{#1}}
\newcommand{\BuiltInTok}[1]{#1}
\newcommand{\CharTok}[1]{\textcolor[rgb]{0.31,0.60,0.02}{#1}}
\newcommand{\CommentTok}[1]{\textcolor[rgb]{0.56,0.35,0.01}{\textit{#1}}}
\newcommand{\CommentVarTok}[1]{\textcolor[rgb]{0.56,0.35,0.01}{\textbf{\textit{#1}}}}
\newcommand{\ConstantTok}[1]{\textcolor[rgb]{0.56,0.35,0.01}{#1}}
\newcommand{\ControlFlowTok}[1]{\textcolor[rgb]{0.13,0.29,0.53}{\textbf{#1}}}
\newcommand{\DataTypeTok}[1]{\textcolor[rgb]{0.13,0.29,0.53}{#1}}
\newcommand{\DecValTok}[1]{\textcolor[rgb]{0.00,0.00,0.81}{#1}}
\newcommand{\DocumentationTok}[1]{\textcolor[rgb]{0.56,0.35,0.01}{\textbf{\textit{#1}}}}
\newcommand{\ErrorTok}[1]{\textcolor[rgb]{0.64,0.00,0.00}{\textbf{#1}}}
\newcommand{\ExtensionTok}[1]{#1}
\newcommand{\FloatTok}[1]{\textcolor[rgb]{0.00,0.00,0.81}{#1}}
\newcommand{\FunctionTok}[1]{\textcolor[rgb]{0.13,0.29,0.53}{\textbf{#1}}}
\newcommand{\ImportTok}[1]{#1}
\newcommand{\InformationTok}[1]{\textcolor[rgb]{0.56,0.35,0.01}{\textbf{\textit{#1}}}}
\newcommand{\KeywordTok}[1]{\textcolor[rgb]{0.13,0.29,0.53}{\textbf{#1}}}
\newcommand{\NormalTok}[1]{#1}
\newcommand{\OperatorTok}[1]{\textcolor[rgb]{0.81,0.36,0.00}{\textbf{#1}}}
\newcommand{\OtherTok}[1]{\textcolor[rgb]{0.56,0.35,0.01}{#1}}
\newcommand{\PreprocessorTok}[1]{\textcolor[rgb]{0.56,0.35,0.01}{\textit{#1}}}
\newcommand{\RegionMarkerTok}[1]{#1}
\newcommand{\SpecialCharTok}[1]{\textcolor[rgb]{0.81,0.36,0.00}{\textbf{#1}}}
\newcommand{\SpecialStringTok}[1]{\textcolor[rgb]{0.31,0.60,0.02}{#1}}
\newcommand{\StringTok}[1]{\textcolor[rgb]{0.31,0.60,0.02}{#1}}
\newcommand{\VariableTok}[1]{\textcolor[rgb]{0.00,0.00,0.00}{#1}}
\newcommand{\VerbatimStringTok}[1]{\textcolor[rgb]{0.31,0.60,0.02}{#1}}
\newcommand{\WarningTok}[1]{\textcolor[rgb]{0.56,0.35,0.01}{\textbf{\textit{#1}}}}
\usepackage{graphicx}
\makeatletter
\def\maxwidth{\ifdim\Gin@nat@width>\linewidth\linewidth\else\Gin@nat@width\fi}
\def\maxheight{\ifdim\Gin@nat@height>\textheight\textheight\else\Gin@nat@height\fi}
\makeatother
% Scale images if necessary, so that they will not overflow the page
% margins by default, and it is still possible to overwrite the defaults
% using explicit options in \includegraphics[width, height, ...]{}
\setkeys{Gin}{width=\maxwidth,height=\maxheight,keepaspectratio}
% Set default figure placement to htbp
\makeatletter
\def\fps@figure{htbp}
\makeatother
\setlength{\emergencystretch}{3em} % prevent overfull lines
\providecommand{\tightlist}{%
  \setlength{\itemsep}{0pt}\setlength{\parskip}{0pt}}
\setcounter{secnumdepth}{-\maxdimen} % remove section numbering
\ifLuaTeX
  \usepackage{selnolig}  % disable illegal ligatures
\fi
\IfFileExists{bookmark.sty}{\usepackage{bookmark}}{\usepackage{hyperref}}
\IfFileExists{xurl.sty}{\usepackage{xurl}}{} % add URL line breaks if available
\urlstyle{same}
\hypersetup{
  pdftitle={Report},
  hidelinks,
  pdfcreator={LaTeX via pandoc}}

\title{Report}
\author{}
\date{\vspace{-2.5em}}

\begin{document}
\maketitle

\hypertarget{introduction}{%
\section{Introduction}\label{introduction}}

\hypertarget{setup}{%
\subsection{1. Setup}\label{setup}}

\begin{Shaded}
\begin{Highlighting}[]
\FunctionTok{library}\NormalTok{(tidyverse)}
\end{Highlighting}
\end{Shaded}

\begin{verbatim}
## -- Attaching core tidyverse packages ------------------------ tidyverse 2.0.0 --
## v dplyr     1.1.4     v readr     2.1.5
## v forcats   1.0.0     v stringr   1.5.1
## v ggplot2   3.5.1     v tibble    3.2.1
## v lubridate 1.9.3     v tidyr     1.3.1
## v purrr     1.0.2     
## -- Conflicts ------------------------------------------ tidyverse_conflicts() --
## x dplyr::filter() masks stats::filter()
## x dplyr::lag()    masks stats::lag()
## i Use the conflicted package (<http://conflicted.r-lib.org/>) to force all conflicts to become errors
\end{verbatim}

\begin{Shaded}
\begin{Highlighting}[]
\FunctionTok{library}\NormalTok{(here)}
\end{Highlighting}
\end{Shaded}

\begin{verbatim}
## here() starts at C:/Users/vopq/OneDrive - University of Cincinnati/Fall 2024/BANA4080
\end{verbatim}

\begin{Shaded}
\begin{Highlighting}[]
\FunctionTok{library}\NormalTok{(lubridate)}
\FunctionTok{library}\NormalTok{(ggplot2)}
\FunctionTok{library}\NormalTok{(readr)}
\FunctionTok{library}\NormalTok{(completejourney)}
\end{Highlighting}
\end{Shaded}

\begin{verbatim}
## Welcome to the completejourney package! Learn more about these data
## sets at http://bit.ly/completejourney.
\end{verbatim}

\begin{Shaded}
\begin{Highlighting}[]
\NormalTok{path }\OtherTok{\textless{}{-}} \FunctionTok{here}\NormalTok{(}\StringTok{\textquotesingle{}BANA4080\_midterm\_project\textquotesingle{}}\NormalTok{, }\StringTok{\textquotesingle{}main\_data\textquotesingle{}}\NormalTok{)}

\NormalTok{display\_location\_labels }\OtherTok{\textless{}{-}} \FunctionTok{c}\NormalTok{(}\StringTok{"0"}\OtherTok{=}\StringTok{"No display"}\NormalTok{, }
                             \StringTok{"1"}\OtherTok{=}\StringTok{"Store front"}\NormalTok{, }
                             \StringTok{"2"}\OtherTok{=}\StringTok{"Store rear"}\NormalTok{, }
                             \StringTok{"3"}\OtherTok{=}\StringTok{"Front end cap"}\NormalTok{, }
                             \StringTok{"4"}\OtherTok{=}\StringTok{"Mid{-}aisle end cap"}\NormalTok{, }
                             \StringTok{"5"}\OtherTok{=}\StringTok{"Rear end cap"}\NormalTok{, }
                             \StringTok{"6"}\OtherTok{=}\StringTok{"Side aisle end cap"}\NormalTok{, }
                             \StringTok{"7"}\OtherTok{=}\StringTok{"In{-}aisle"}\NormalTok{, }
                             \StringTok{"9"}\OtherTok{=}\StringTok{"Secondary location display"}\NormalTok{, }
                             \StringTok{"A"}\OtherTok{=}\StringTok{"In{-}shelf"}\NormalTok{)}

\NormalTok{mailer\_location\_labels }\OtherTok{\textless{}{-}} \FunctionTok{c}\NormalTok{(}\StringTok{"0"}\OtherTok{=}\StringTok{"Not on ad"}\NormalTok{, }
                            \StringTok{"A"}\OtherTok{=}\StringTok{"Interior page feature"}\NormalTok{, }
                            \StringTok{"C"}\OtherTok{=}\StringTok{"Interior page line item"}\NormalTok{, }
                            \StringTok{"D"}\OtherTok{=}\StringTok{"Front page feature"}\NormalTok{, }
                            \StringTok{"F"}\OtherTok{=}\StringTok{"Back page feature"}\NormalTok{, }
                            \StringTok{"H"}\OtherTok{=}\StringTok{"Wrap front feature"}\NormalTok{, }
                            \StringTok{"J"}\OtherTok{=}\StringTok{"Wrap interior coupon"}\NormalTok{, }
                            \StringTok{"L"}\OtherTok{=}\StringTok{"Wrap back feature"}\NormalTok{, }
                            \StringTok{"P"}\OtherTok{=}\StringTok{"Interior page coupon"}\NormalTok{, }
                            \StringTok{"X"}\OtherTok{=}\StringTok{"Free on interior page"}\NormalTok{, }
                            \StringTok{"Z"}\OtherTok{=}\StringTok{"Free on front page/back page/or wrap"}\NormalTok{)}

\FunctionTok{c}\NormalTok{(promotions, transactions) }\SpecialCharTok{\%\textless{}{-}\%} \FunctionTok{get\_data}\NormalTok{(}\AttributeTok{which =} \StringTok{\textquotesingle{}both\textquotesingle{}}\NormalTok{, }\AttributeTok{verbose =} \ConstantTok{FALSE}\NormalTok{)}
\NormalTok{transactions}\SpecialCharTok{$}\NormalTok{product\_id }\OtherTok{\textless{}{-}} \FunctionTok{as.double}\NormalTok{(transactions}\SpecialCharTok{$}\NormalTok{product\_id)}
\end{Highlighting}
\end{Shaded}

\hypertarget{load-datasets}{%
\subsection{2. Load datasets}\label{load-datasets}}

\begin{Shaded}
\begin{Highlighting}[]
\NormalTok{full\_transactions }\OtherTok{\textless{}{-}} \FunctionTok{read\_csv}\NormalTok{(}\FunctionTok{here}\NormalTok{(path, }\StringTok{"Final\_Transactions\_With\_Redemptions.csv"}\NormalTok{))}
\end{Highlighting}
\end{Shaded}

\begin{verbatim}
## Rows: 2585 Columns: 11
## -- Column specification --------------------------------------------------------
## Delimiter: ","
## chr  (1): campaign_type
## dbl  (7): coupon_upc, product_id, store_id, campaign_id, household_id, sales...
## date (3): start_date, end_date, date
## 
## i Use `spec()` to retrieve the full column specification for this data.
## i Specify the column types or set `show_col_types = FALSE` to quiet this message.
\end{verbatim}

\begin{Shaded}
\begin{Highlighting}[]
\NormalTok{trans\_w\_display }\OtherTok{\textless{}{-}} \FunctionTok{read\_csv}\NormalTok{(}\FunctionTok{here}\NormalTok{(path, }\StringTok{"Transactions\_W\_Redemptions\_Display\_Location.csv"}\NormalTok{))}
\end{Highlighting}
\end{Shaded}

\begin{verbatim}
## Rows: 1931 Columns: 7
## -- Column specification --------------------------------------------------------
## Delimiter: ","
## chr (4): display_location, department, product_category, product_type
## dbl (3): product_id, store_id, total_sales
## 
## i Use `spec()` to retrieve the full column specification for this data.
## i Specify the column types or set `show_col_types = FALSE` to quiet this message.
\end{verbatim}

\begin{Shaded}
\begin{Highlighting}[]
\NormalTok{trans\_w\_mailer }\OtherTok{\textless{}{-}} \FunctionTok{read\_csv}\NormalTok{(}\FunctionTok{here}\NormalTok{(path, }\StringTok{"Transactions\_W\_Redemptions\_Mailer\_Location.csv"}\NormalTok{))}
\end{Highlighting}
\end{Shaded}

\begin{verbatim}
## Rows: 3934 Columns: 7
## -- Column specification --------------------------------------------------------
## Delimiter: ","
## chr (4): mailer_location, department, product_category, product_type
## dbl (3): product_id, store_id, total_sales
## 
## i Use `spec()` to retrieve the full column specification for this data.
## i Specify the column types or set `show_col_types = FALSE` to quiet this message.
\end{verbatim}

\begin{Shaded}
\begin{Highlighting}[]
\NormalTok{trans\_w\_display0 }\OtherTok{\textless{}{-}} \FunctionTok{read\_csv}\NormalTok{(}\FunctionTok{here}\NormalTok{(path, }\StringTok{"Transactions\_W\_Redemptions\_Not\_Displayed.csv"}\NormalTok{))}
\end{Highlighting}
\end{Shaded}

\begin{verbatim}
## Rows: 1931 Columns: 6
## -- Column specification --------------------------------------------------------
## Delimiter: ","
## chr (3): department, product_category, product_type
## dbl (3): product_id, store_id, total_sales
## 
## i Use `spec()` to retrieve the full column specification for this data.
## i Specify the column types or set `show_col_types = FALSE` to quiet this message.
\end{verbatim}

\begin{Shaded}
\begin{Highlighting}[]
\NormalTok{trans\_w\_mailer0 }\OtherTok{\textless{}{-}} \FunctionTok{read\_csv}\NormalTok{(}\FunctionTok{here}\NormalTok{(path, }\StringTok{"Transactions\_W\_Redemptions\_Not\_Mailed.csv"}\NormalTok{))}
\end{Highlighting}
\end{Shaded}

\begin{verbatim}
## Rows: 816 Columns: 6
## -- Column specification --------------------------------------------------------
## Delimiter: ","
## chr (3): department, product_category, product_type
## dbl (3): product_id, store_id, total_sales
## 
## i Use `spec()` to retrieve the full column specification for this data.
## i Specify the column types or set `show_col_types = FALSE` to quiet this message.
\end{verbatim}

\hypertarget{hypothesis-different-coupon-types-affect-sales-differently}{%
\subsection{3. Hypothesis: Different coupon types affect sales
differently}\label{hypothesis-different-coupon-types-affect-sales-differently}}

\begin{itemize}
\tightlist
\item
  Assuming `campaign\_type' contains different coupon types
\end{itemize}

\begin{Shaded}
\begin{Highlighting}[]
\NormalTok{summary\_transactions }\OtherTok{\textless{}{-}}\NormalTok{ full\_transactions }\SpecialCharTok{\%\textgreater{}\%}
  \FunctionTok{group\_by}\NormalTok{(campaign\_type) }\SpecialCharTok{\%\textgreater{}\%}
  \FunctionTok{summarise}\NormalTok{(}
    \AttributeTok{total\_sales =} \FunctionTok{sum}\NormalTok{(sales\_value),}
    \AttributeTok{count\_type =} \FunctionTok{n}\NormalTok{()}
\NormalTok{)}

\FunctionTok{ggplot}\NormalTok{(}\AttributeTok{data =}\NormalTok{ summary\_transactions, }\FunctionTok{aes}\NormalTok{(}\AttributeTok{x =}\NormalTok{ campaign\_type)) }\SpecialCharTok{+} 
  \FunctionTok{geom\_col}\NormalTok{(}\FunctionTok{aes}\NormalTok{(}\AttributeTok{y =}\NormalTok{ count\_type, }\AttributeTok{fill =} \StringTok{\textquotesingle{}Count of campaign type\textquotesingle{}}\NormalTok{), }\AttributeTok{width =} \FloatTok{0.5}\NormalTok{) }\SpecialCharTok{+}
  \FunctionTok{geom\_line}\NormalTok{(}\FunctionTok{aes}\NormalTok{(}\AttributeTok{y =}\NormalTok{ total\_sales, }\AttributeTok{color =} \StringTok{\textquotesingle{}Total Sales\textquotesingle{}}\NormalTok{), }\AttributeTok{group =} \DecValTok{1}\NormalTok{) }\SpecialCharTok{+}
  \FunctionTok{geom\_point}\NormalTok{(}\FunctionTok{aes}\NormalTok{(}\AttributeTok{y =}\NormalTok{ total\_sales, }\AttributeTok{color =} \StringTok{\textquotesingle{}Total Sales\textquotesingle{}}\NormalTok{), }\AttributeTok{size =} \FloatTok{3.0}\NormalTok{) }\SpecialCharTok{+} 
  \FunctionTok{scale\_y\_continuous}\NormalTok{(}
    \AttributeTok{name =} \StringTok{"Count of campaign type"}\NormalTok{,}
    \AttributeTok{breaks =} \FunctionTok{seq}\NormalTok{(}\DecValTok{0}\NormalTok{, }\FunctionTok{max}\NormalTok{(summary\_transactions}\SpecialCharTok{$}\NormalTok{total\_sales) }\SpecialCharTok{+} \DecValTok{1}\NormalTok{, }\AttributeTok{by =} \DecValTok{400}\NormalTok{),}
    \AttributeTok{sec.axis =} \FunctionTok{sec\_axis}\NormalTok{(}\SpecialCharTok{\textasciitilde{}}\NormalTok{., }
                        \AttributeTok{name =} \StringTok{"Total Sales"}\NormalTok{,}
                        \AttributeTok{breaks =} \FunctionTok{seq}\NormalTok{(}\DecValTok{0}\NormalTok{, }\FunctionTok{max}\NormalTok{(summary\_transactions}\SpecialCharTok{$}\NormalTok{total\_sales) }\SpecialCharTok{+} \DecValTok{1}\NormalTok{, }\AttributeTok{by =} \DecValTok{400}\NormalTok{))}
\NormalTok{  ) }\SpecialCharTok{+}
  \FunctionTok{labs}\NormalTok{(}\AttributeTok{x =} \StringTok{\textquotesingle{}Campaign Types\textquotesingle{}}\NormalTok{, }\AttributeTok{title =} \StringTok{"Total Sales Value and Count of Redemptions Per Campaign Type"}\NormalTok{) }\SpecialCharTok{+} 
  \FunctionTok{scale\_fill\_manual}\NormalTok{(}\AttributeTok{name =} \StringTok{\textquotesingle{}\textquotesingle{}}\NormalTok{, }\AttributeTok{values =} \FunctionTok{c}\NormalTok{(}\StringTok{"Count of campaign type"} \OtherTok{=} \StringTok{"lightblue"}\NormalTok{)) }\SpecialCharTok{+}
  \FunctionTok{scale\_color\_manual}\NormalTok{(}\AttributeTok{name =} \StringTok{\textquotesingle{}\textquotesingle{}}\NormalTok{, }\AttributeTok{values =} \FunctionTok{c}\NormalTok{(}\StringTok{"Total Sales"} \OtherTok{=} \StringTok{\textquotesingle{}red\textquotesingle{}}\NormalTok{)) }\SpecialCharTok{+} 
  \FunctionTok{theme\_minimal}\NormalTok{() }\SpecialCharTok{+} 
  \FunctionTok{theme}\NormalTok{(}\AttributeTok{legend.position =} \StringTok{\textquotesingle{}bottom\textquotesingle{}}\NormalTok{,}
        \AttributeTok{axis.title.x =} \FunctionTok{element\_text}\NormalTok{(}\AttributeTok{margin =} \FunctionTok{margin}\NormalTok{(}\AttributeTok{t =} \DecValTok{10}\NormalTok{)),}
        \AttributeTok{axis.title.y.left =} \FunctionTok{element\_text}\NormalTok{(}\AttributeTok{margin =} \FunctionTok{margin}\NormalTok{(}\AttributeTok{r =} \DecValTok{10}\NormalTok{)),}
        \AttributeTok{axis.title.y.right =} \FunctionTok{element\_text}\NormalTok{(}\AttributeTok{margin =} \FunctionTok{margin}\NormalTok{(}\AttributeTok{l =} \DecValTok{10}\NormalTok{)),}
\NormalTok{        ) }\SpecialCharTok{+}
  \FunctionTok{annotate}\NormalTok{(}\StringTok{"text"}\NormalTok{, }\AttributeTok{x =} \StringTok{"Type A"}\NormalTok{, }\AttributeTok{y =} \DecValTok{7603}\NormalTok{, }\AttributeTok{label =} \StringTok{"Exceptionally High"}\NormalTok{, }\AttributeTok{color =} \StringTok{"blue"}\NormalTok{, }\AttributeTok{size =} \DecValTok{6}\NormalTok{, }\AttributeTok{vjust =} \SpecialCharTok{{-}}\DecValTok{1}\NormalTok{)}
\end{Highlighting}
\end{Shaded}

\includegraphics{Report_files/figure-latex/unnamed-chunk-3-1.pdf}

Thoughts:

\begin{itemize}
\tightlist
\item
  Type A has significantly higher redemption counts and sales value
  =\textgreater{} might have more look into what type A is, how it
  works, which products that type A works on most.
\end{itemize}

\begin{Shaded}
\begin{Highlighting}[]
\NormalTok{products}\SpecialCharTok{$}\NormalTok{product\_id }\OtherTok{\textless{}{-}} \FunctionTok{as.double}\NormalTok{(products}\SpecialCharTok{$}\NormalTok{product\_id)}

\NormalTok{sample }\OtherTok{\textless{}{-}}\NormalTok{ full\_transactions }\SpecialCharTok{\%\textgreater{}\%}
  \FunctionTok{group\_by}\NormalTok{(product\_id, campaign\_type) }\SpecialCharTok{\%\textgreater{}\%}
  \FunctionTok{summarise}\NormalTok{(}\AttributeTok{total =} \FunctionTok{sum}\NormalTok{(sales\_value), }\AttributeTok{.groups=}\StringTok{"drop"}\NormalTok{) }\SpecialCharTok{\%\textgreater{}\%}
  \FunctionTok{left\_join}\NormalTok{(products, }\AttributeTok{by=}\StringTok{"product\_id"}\NormalTok{) }\SpecialCharTok{\%\textgreater{}\%}
  \FunctionTok{select}\NormalTok{(campaign\_type, department, total) }\SpecialCharTok{\%\textgreater{}\%}
  \FunctionTok{group\_by}\NormalTok{(campaign\_type, department) }\SpecialCharTok{\%\textgreater{}\%}
  \FunctionTok{summarise}\NormalTok{(}\AttributeTok{total\_new =} \FunctionTok{sum}\NormalTok{(total), }\AttributeTok{.groups=}\StringTok{"drop"}\NormalTok{)}
  
\FunctionTok{ggplot}\NormalTok{(}\AttributeTok{data =}\NormalTok{ sample, }\FunctionTok{aes}\NormalTok{(}\AttributeTok{x =}\NormalTok{ campaign\_type, }\AttributeTok{y =}\NormalTok{ department, }\AttributeTok{fill =}\NormalTok{ total\_new)) }\SpecialCharTok{+}
  \FunctionTok{geom\_tile}\NormalTok{() }\SpecialCharTok{+}
  \FunctionTok{scale\_fill\_gradient2}\NormalTok{(}\AttributeTok{low=}\StringTok{"white"}\NormalTok{, }\AttributeTok{mid=}\StringTok{"lightblue"}\NormalTok{, }\AttributeTok{high=}\StringTok{"blue"}\NormalTok{, }\AttributeTok{name =} \StringTok{"Sales Value"}\NormalTok{) }\SpecialCharTok{+}
  \FunctionTok{geom\_text}\NormalTok{(}\FunctionTok{aes}\NormalTok{(}\AttributeTok{label=}\FunctionTok{round}\NormalTok{(total\_new, }\DecValTok{2}\NormalTok{)), }\AttributeTok{color=}\StringTok{"black"}\NormalTok{, }\AttributeTok{size=}\DecValTok{4}\NormalTok{) }\SpecialCharTok{+}
  \FunctionTok{theme\_minimal}\NormalTok{() }\SpecialCharTok{+}
  \FunctionTok{labs}\NormalTok{(}\AttributeTok{title =} \StringTok{"Distribution of Sales Value by Campaign Types and Departments"}\NormalTok{,}
       \AttributeTok{x =} \StringTok{"Campaign Types"}\NormalTok{,}
       \AttributeTok{y =} \StringTok{"Deparments"}\NormalTok{) }\SpecialCharTok{+}
  \FunctionTok{theme}\NormalTok{(}\AttributeTok{axis.title.x =} \FunctionTok{element\_text}\NormalTok{(}\AttributeTok{margin =} \FunctionTok{margin}\NormalTok{(}\AttributeTok{t =} \DecValTok{10}\NormalTok{)),}
        \AttributeTok{axis.title.y =} \FunctionTok{element\_text}\NormalTok{(}\AttributeTok{margin =} \FunctionTok{margin}\NormalTok{(}\AttributeTok{r =} \DecValTok{10}\NormalTok{)),}
\NormalTok{)}
\end{Highlighting}
\end{Shaded}

\includegraphics{Report_files/figure-latex/unnamed-chunk-4-1.pdf} -
Thoughts:

\begin{itemize}
\tightlist
\item
  The Grocery department stays on top of total sales value regardless of
  campaign types. This can be explained by being daily necessities. In
  the following EDA, we can omit Grocery department to see how other
  departments are doing.
\item
  Type A campaign has coupons redeemed mostly and in every department.
\end{itemize}

\hypertarget{stuff-at-certain-display-locations-with-coupons-will-have-the-most-purchase.}{%
\subsection{5. Stuff at certain display locations with coupons will have
the most
purchase.}\label{stuff-at-certain-display-locations-with-coupons-will-have-the-most-purchase.}}

\begin{itemize}
\tightlist
\item
  As mentioned above, we exclude Grocery department in this
  visualization.
\end{itemize}

\begin{Shaded}
\begin{Highlighting}[]
\NormalTok{sample }\OtherTok{\textless{}{-}}\NormalTok{ trans\_w\_display }\SpecialCharTok{\%\textgreater{}\%}
\NormalTok{  dplyr}\SpecialCharTok{::}\FunctionTok{filter}\NormalTok{(department }\SpecialCharTok{!=} \StringTok{"GROCERY"}\NormalTok{) }\SpecialCharTok{\%\textgreater{}\%}
  \FunctionTok{group\_by}\NormalTok{(display\_location, department) }\SpecialCharTok{\%\textgreater{}\%}
  \FunctionTok{summarize}\NormalTok{(}\AttributeTok{total =} \FunctionTok{sum}\NormalTok{(total\_sales), }\AttributeTok{.groups =} \StringTok{\textquotesingle{}drop\textquotesingle{}}\NormalTok{)}

\FunctionTok{print}\NormalTok{(sample, }\AttributeTok{n =} \DecValTok{5}\NormalTok{)}
\end{Highlighting}
\end{Shaded}

\begin{verbatim}
## # A tibble: 45 x 3
##   display_location department total
##   <chr>            <chr>      <dbl>
## 1 1                DELI        3.99
## 2 1                DRUG GM    87.4 
## 3 1                MEAT-PCKGD  6.5 
## 4 1                NUTRITION   7   
## 5 2                DRUG GM    39.9 
## # i 40 more rows
\end{verbatim}

\hypertarget{a.-stack-bar-plot-with-raw-sales-value-for-each-display-location}{%
\subsubsection{a. Stack bar plot with raw sales value for each display
location}\label{a.-stack-bar-plot-with-raw-sales-value-for-each-display-location}}

\begin{Shaded}
\begin{Highlighting}[]
\FunctionTok{ggplot}\NormalTok{(}\AttributeTok{data =}\NormalTok{ sample, }\FunctionTok{aes}\NormalTok{(}\AttributeTok{x =}\NormalTok{ display\_location, }\AttributeTok{y =}\NormalTok{ total, }\AttributeTok{fill =}\NormalTok{ department)) }\SpecialCharTok{+} 
  \FunctionTok{geom\_bar}\NormalTok{(}\AttributeTok{stat =} \StringTok{"identity"}\NormalTok{, }\AttributeTok{position =} \StringTok{"stack"}\NormalTok{) }\SpecialCharTok{+}
  \FunctionTok{scale\_fill\_manual}\NormalTok{(}\AttributeTok{values =} \FunctionTok{c}\NormalTok{(}\StringTok{"DELI"} \OtherTok{=} \StringTok{"\#FF9999"}\NormalTok{, }
                               \StringTok{"DRUG GM"} \OtherTok{=} \StringTok{"\#66CC99"}\NormalTok{, }
                               \StringTok{"MEAT{-}PCKGD"} \OtherTok{=} \StringTok{"\#FFCC00"}\NormalTok{, }
                               \StringTok{"NUTRITION"} \OtherTok{=} \StringTok{"\#3399FF"}\NormalTok{, }
                               \StringTok{"MEAT"} \OtherTok{=} \StringTok{"\#FF66CC"}\NormalTok{, }
                               \StringTok{"PASTRY"} \OtherTok{=} \StringTok{"\#FF9966"}\NormalTok{, }
                               \StringTok{"SEAFOOD{-}PCKGD"} \OtherTok{=} \StringTok{"\#99CCFF"}\NormalTok{)) }\SpecialCharTok{+}
  \FunctionTok{scale\_x\_discrete}\NormalTok{(}\AttributeTok{labels =}\NormalTok{ display\_location\_labels) }\SpecialCharTok{+}
  \FunctionTok{theme\_minimal}\NormalTok{() }\SpecialCharTok{+}
  \FunctionTok{labs}\NormalTok{(}\AttributeTok{title =} \StringTok{"Total Sales by Department and Campaign Type in Percentage"}\NormalTok{,}
       \AttributeTok{x =} \StringTok{"Display Locations"}\NormalTok{,}
       \AttributeTok{y =} \StringTok{"Sales Value"}\NormalTok{,}
       \AttributeTok{fill =} \StringTok{"Departments"}\NormalTok{) }\SpecialCharTok{+}
  \FunctionTok{theme}\NormalTok{(}\AttributeTok{axis.title.x =} \FunctionTok{element\_text}\NormalTok{(}\AttributeTok{margin =} \FunctionTok{margin}\NormalTok{(}\AttributeTok{t =} \DecValTok{10}\NormalTok{)),}
        \AttributeTok{axis.title.y =} \FunctionTok{element\_text}\NormalTok{(}\AttributeTok{margin =} \FunctionTok{margin}\NormalTok{(}\AttributeTok{r =} \DecValTok{10}\NormalTok{)),}
\NormalTok{)}
\end{Highlighting}
\end{Shaded}

\includegraphics{Report_files/figure-latex/unnamed-chunk-6-1.pdf}

\hypertarget{b.-plot-with-sales-value-being-normalize-to-percentage}{%
\subsubsection{b. Plot with sales value being normalize to
percentage}\label{b.-plot-with-sales-value-being-normalize-to-percentage}}

\begin{Shaded}
\begin{Highlighting}[]
\NormalTok{summary\_df }\OtherTok{\textless{}{-}}\NormalTok{ sample }\SpecialCharTok{\%\textgreater{}\%}
  \FunctionTok{group\_by}\NormalTok{(display\_location, department) }\SpecialCharTok{\%\textgreater{}\%}
  \FunctionTok{summarise}\NormalTok{(}\AttributeTok{total\_sales =} \FunctionTok{sum}\NormalTok{(total), }\AttributeTok{.groups =} \StringTok{"drop"}\NormalTok{)}

\NormalTok{total\_sales\_per\_display\_location }\OtherTok{\textless{}{-}}\NormalTok{ summary\_df }\SpecialCharTok{\%\textgreater{}\%}
  \FunctionTok{group\_by}\NormalTok{(display\_location) }\SpecialCharTok{\%\textgreater{}\%}
  \FunctionTok{summarise}\NormalTok{(}\AttributeTok{total\_sales\_location =} \FunctionTok{sum}\NormalTok{(total\_sales), }\AttributeTok{.groups =} \StringTok{"drop"}\NormalTok{)}

\NormalTok{final\_df }\OtherTok{\textless{}{-}}\NormalTok{ summary\_df }\SpecialCharTok{\%\textgreater{}\%}
  \FunctionTok{left\_join}\NormalTok{(total\_sales\_per\_display\_location, }\AttributeTok{by =} \StringTok{"display\_location"}\NormalTok{) }\SpecialCharTok{\%\textgreater{}\%}
  \FunctionTok{mutate}\NormalTok{(}\AttributeTok{percentage =} \FunctionTok{round}\NormalTok{((total\_sales }\SpecialCharTok{/}\NormalTok{ total\_sales\_location) }\SpecialCharTok{*} \DecValTok{100}\NormalTok{,}\DecValTok{2}\NormalTok{))}

\FunctionTok{ggplot}\NormalTok{(final\_df, }\FunctionTok{aes}\NormalTok{(}\AttributeTok{x =}\NormalTok{ display\_location, }\AttributeTok{y =}\NormalTok{ percentage, }\AttributeTok{fill =}\NormalTok{ department)) }\SpecialCharTok{+}
  \FunctionTok{geom\_bar}\NormalTok{(}\AttributeTok{stat =} \StringTok{"identity"}\NormalTok{, }\AttributeTok{position =} \StringTok{"stack"}\NormalTok{) }\SpecialCharTok{+}
  \FunctionTok{scale\_fill\_manual}\NormalTok{(}\AttributeTok{values =} \FunctionTok{c}\NormalTok{(}\StringTok{"DELI"} \OtherTok{=} \StringTok{"\#FF9999"}\NormalTok{, }
                               \StringTok{"DRUG GM"} \OtherTok{=} \StringTok{"\#66CC99"}\NormalTok{, }
                               \StringTok{"MEAT{-}PCKGD"} \OtherTok{=} \StringTok{"\#FFCC00"}\NormalTok{, }
                               \StringTok{"NUTRITION"} \OtherTok{=} \StringTok{"\#3399FF"}\NormalTok{, }
                               \StringTok{"MEAT"} \OtherTok{=} \StringTok{"\#FF66CC"}\NormalTok{, }
                               \StringTok{"PASTRY"} \OtherTok{=} \StringTok{"\#FF9966"}\NormalTok{, }
                               \StringTok{"SEAFOOD{-}PCKGD"} \OtherTok{=} \StringTok{"\#99CCFF"}\NormalTok{)) }\SpecialCharTok{+}
  \FunctionTok{scale\_x\_discrete}\NormalTok{(}\AttributeTok{labels =}\NormalTok{ display\_location\_labels) }\SpecialCharTok{+}
  \FunctionTok{theme\_minimal}\NormalTok{() }\SpecialCharTok{+}
  \FunctionTok{labs}\NormalTok{(}\AttributeTok{title =} \StringTok{"Total Sales by Department and Campaign Type in Percentage"}\NormalTok{,}
       \AttributeTok{x =} \StringTok{"Display Locations"}\NormalTok{,}
       \AttributeTok{y =} \StringTok{"Percentage Sales"}\NormalTok{,}
       \AttributeTok{fill =} \StringTok{"Departments"}\NormalTok{) }\SpecialCharTok{+}
  \FunctionTok{theme}\NormalTok{(}\AttributeTok{axis.title.x =} \FunctionTok{element\_text}\NormalTok{(}\AttributeTok{margin =} \FunctionTok{margin}\NormalTok{(}\AttributeTok{t =} \DecValTok{10}\NormalTok{)),}
        \AttributeTok{axis.title.y =} \FunctionTok{element\_text}\NormalTok{(}\AttributeTok{margin =} \FunctionTok{margin}\NormalTok{(}\AttributeTok{r =} \DecValTok{10}\NormalTok{)),}
\NormalTok{)}
\end{Highlighting}
\end{Shaded}

\includegraphics{Report_files/figure-latex/unnamed-chunk-7-1.pdf} -
Thoughts: + Drug GM seems to have the most sales after Grocery
department

\hypertarget{comparisons-between-being-displayed-and-not-displayed---having-coupons-redeemed}{%
\subsection{6. Comparisons between being displayed and not displayed -
Having coupons
redeemed}\label{comparisons-between-being-displayed-and-not-displayed---having-coupons-redeemed}}

\hypertarget{display}{%
\subsubsection{Display}\label{display}}

\begin{Shaded}
\begin{Highlighting}[]
\NormalTok{sample1 }\OtherTok{\textless{}{-}}\NormalTok{ trans\_w\_display }\SpecialCharTok{\%\textgreater{}\%}
  \FunctionTok{group\_by}\NormalTok{(department) }\SpecialCharTok{\%\textgreater{}\%}
  \FunctionTok{summarize}\NormalTok{(}\AttributeTok{total =} \FunctionTok{sum}\NormalTok{(total\_sales), }\AttributeTok{.groups =} \StringTok{\textquotesingle{}drop\textquotesingle{}}\NormalTok{) }\SpecialCharTok{\%\textgreater{}\%}
  \FunctionTok{arrange}\NormalTok{(}\FunctionTok{desc}\NormalTok{(department))}

\FunctionTok{ggplot}\NormalTok{(}\AttributeTok{data =}\NormalTok{ sample1, }\FunctionTok{aes}\NormalTok{(}\AttributeTok{x =}\NormalTok{ department, }\AttributeTok{y =}\NormalTok{ total)) }\SpecialCharTok{+}
  \FunctionTok{geom\_bar}\NormalTok{(}\AttributeTok{stat =} \StringTok{"identity"}\NormalTok{, }\AttributeTok{fill =} \StringTok{"lightblue"}\NormalTok{) }\SpecialCharTok{+}
  \FunctionTok{coord\_flip}\NormalTok{() }\SpecialCharTok{+} 
  \FunctionTok{scale\_y\_continuous}\NormalTok{(}\AttributeTok{breaks =} \FunctionTok{seq}\NormalTok{(}\DecValTok{0}\NormalTok{, }\FunctionTok{max}\NormalTok{(sample1}\SpecialCharTok{$}\NormalTok{total)}\SpecialCharTok{+}\DecValTok{1}\NormalTok{, }\AttributeTok{by=}\DecValTok{200}\NormalTok{)) }\SpecialCharTok{+}
  \FunctionTok{geom\_text}\NormalTok{(}\FunctionTok{aes}\NormalTok{(}\AttributeTok{label =} \FunctionTok{round}\NormalTok{(total, }\DecValTok{2}\NormalTok{)),}
            \AttributeTok{hjust =} \SpecialCharTok{{-}}\FloatTok{0.1}\NormalTok{,}
            \AttributeTok{color =} \StringTok{"black"}\NormalTok{) }\SpecialCharTok{+}
  \FunctionTok{labs}\NormalTok{(}\AttributeTok{title =} \StringTok{"Total Sales Across Department for Being Displayed"}\NormalTok{,}
       \AttributeTok{x =} \StringTok{"Departments"}\NormalTok{,}
       \AttributeTok{y =} \StringTok{"Total Sales"}\NormalTok{) }\SpecialCharTok{+}
  \FunctionTok{theme\_minimal}\NormalTok{() }\SpecialCharTok{+} 
  \FunctionTok{theme}\NormalTok{(}\AttributeTok{axis.title.x =} \FunctionTok{element\_text}\NormalTok{(}\AttributeTok{margin =} \FunctionTok{margin}\NormalTok{(}\AttributeTok{t =} \DecValTok{10}\NormalTok{)))}
\end{Highlighting}
\end{Shaded}

\includegraphics{Report_files/figure-latex/unnamed-chunk-8-1.pdf}

\hypertarget{not-display}{%
\subsubsection{Not Display}\label{not-display}}

\begin{Shaded}
\begin{Highlighting}[]
\NormalTok{sample2 }\OtherTok{\textless{}{-}}\NormalTok{ trans\_w\_display0 }\SpecialCharTok{\%\textgreater{}\%}
  \FunctionTok{group\_by}\NormalTok{(department) }\SpecialCharTok{\%\textgreater{}\%}
  \FunctionTok{summarise}\NormalTok{(}\AttributeTok{total =} \FunctionTok{sum}\NormalTok{(total\_sales)) }\SpecialCharTok{\%\textgreater{}\%}
  \FunctionTok{arrange}\NormalTok{(}\FunctionTok{desc}\NormalTok{(department))}

\FunctionTok{ggplot}\NormalTok{(}\AttributeTok{data =}\NormalTok{ sample2, }\FunctionTok{aes}\NormalTok{(}\AttributeTok{x =}\NormalTok{ department, }\AttributeTok{y =}\NormalTok{ total)) }\SpecialCharTok{+}
  \FunctionTok{geom\_bar}\NormalTok{(}\AttributeTok{stat =} \StringTok{"identity"}\NormalTok{, }\AttributeTok{fill =} \StringTok{"lightblue"}\NormalTok{) }\SpecialCharTok{+}
  \FunctionTok{coord\_flip}\NormalTok{() }\SpecialCharTok{+} 
  \FunctionTok{scale\_y\_continuous}\NormalTok{(}\AttributeTok{breaks =} \FunctionTok{seq}\NormalTok{(}\DecValTok{0}\NormalTok{, }\FunctionTok{max}\NormalTok{(sample2}\SpecialCharTok{$}\NormalTok{total)}\SpecialCharTok{+}\DecValTok{1}\NormalTok{, }\AttributeTok{by=}\DecValTok{200}\NormalTok{)) }\SpecialCharTok{+}
  \FunctionTok{geom\_text}\NormalTok{(}\FunctionTok{aes}\NormalTok{(}\AttributeTok{label =} \FunctionTok{round}\NormalTok{(total, }\DecValTok{2}\NormalTok{)),}
            \AttributeTok{hjust =} \SpecialCharTok{{-}}\FloatTok{0.1}\NormalTok{,}
            \AttributeTok{color =} \StringTok{"black"}\NormalTok{) }\SpecialCharTok{+}
  \FunctionTok{labs}\NormalTok{(}\AttributeTok{title =} \StringTok{"Total Sales Across Department for Not Being Displayed"}\NormalTok{,}
       \AttributeTok{x =} \StringTok{"Departments"}\NormalTok{,}
       \AttributeTok{y =} \StringTok{"Total Sales"}\NormalTok{) }\SpecialCharTok{+}
  \FunctionTok{theme\_minimal}\NormalTok{() }\SpecialCharTok{+} 
  \FunctionTok{theme}\NormalTok{(}\AttributeTok{axis.title.x =} \FunctionTok{element\_text}\NormalTok{(}\AttributeTok{margin =} \FunctionTok{margin}\NormalTok{(}\AttributeTok{t =} \DecValTok{10}\NormalTok{)))}
\end{Highlighting}
\end{Shaded}

\includegraphics{Report_files/figure-latex/unnamed-chunk-9-1.pdf} -
Thoughts: + Even without being displayed, Grocery department still has
the highest sales value, following by Drug GM department. However, Meat
Packaged is no longer at third position. This can implicit that people
tends to buy packaged products if being displayed, otherwise people buy
fresh meat.

\end{document}
